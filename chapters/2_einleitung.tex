\chapter{Einleitung}
%Ziel des Versuchs.
\section{Ziel des Versuchs}
Ziel dieses Praktikumsversuchs ist die Untersuchung der Lichtgeschwindigkeit in Luft und die Evaluation des Messverfahrens.
%
\section{Gleichungen}
Der Tabellenwert für die Ausbreitung des Lichtes im Vakuum beträgt $ c_0 = \SI{299792,458}{\frac{km}{s}} $ \cite{Haberle.2007}.\par
Da der Versuch jedoch in Luft durchgeführt wird gilt mit dem Brechungsgesetz und einem Brechungsindex für Luft bei
Standarddruck und -temperatur mit $ n_{Luft}=1,00029 $ \cite{Halliday.2005}:
\begin{equation}
    c = \frac{c_0}{n_{Luft}} = \frac{\SI{299792,458}{\frac{km}{s}}}{1,00029} = \SI{299705,543}{\frac{km}{s}}
    \label{eq:brechung_luft}
\end{equation}
Der zu \(n_{Luft}\) reziproke Wert kann also auch als prozentuale Verlangsamung der Vakuumgeschwindigkeit des Lichtes
verstanden werden.\par\medskip
%
Der im nachfolgenden Kapitel beschriebene Versuchsaufbau verwendet innerhalb der optischen Strecke eine \textsc{Fresnel}-Linse.
Der Abstand der Linsenhauptebene zum Brennpunkt (Fokus) wird Brennweite genannt und aus der Abbildungsgleichung hervorgehend
wie folgt beschrieben:
\begin{equation}
    \frac{1}{f} = \frac{1}{b}+\frac{1}{g} \quad \Leftrightarrow \quad f = \left(\frac{1}{b}+\frac{1}{g}\right)^{-1}=\frac{b g}{b + g}
    \label{eq:brenn}
\end{equation}