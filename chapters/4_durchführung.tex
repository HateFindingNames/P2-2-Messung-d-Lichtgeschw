\chapter{Versuchsdurchführung}
Die Pulsdauer beträgt nur einigen Nanosekunden und ist damit extrem kurz. Durch die schnelle Abfolge der Einzelpulse erscheint
der Strahl für das Auge allerdings kontinuierlich und lässt sich so gut justieren.\par
Zunächst wird der Prismen-Reflektor in größtmöglicher Distanz positioniert. Nachdem die Strahlquelle und das Oszilloskop
eingeschaltet sind können Strahlquelle, \textsc{Fresnel}-Linse und Reflektorspiegel zunächst per Augenmaß so eingestellt werden,
dass auf dem Oszilloskop ein Signal sichtbar wird. Das Referenzsignal sollte hierbei dauerhaft und unverändert auf dem gewählten
Kanal erscheinen. Ist ein Ausschlag auf dem übrigen Kanal erkennbar wird durch Änderung des Abstandes der \textsc{Fresnel}-Linse
zur Strahlquelle die Amplitude des Messignals maximiert.\par
Wenn beide Signale gut erkennbar sind kann die Laufzeitdifferenz \(\Delta t\) auf dem Oszilloskop wie in \bild{subfig:delta_t}
abgelesen und notiert werden. Für die nachfolgenden Messungen wird der Abstand des Prismen-Reflektors so weit schrittweise um \SI{0,5}{m}
verringert, der Abstand der \textsc{Fresnel}-Linse nachjustiert und \(\Delta t\) aufgezeichnet, bis das räumliche Minimum erreicht ist.
%