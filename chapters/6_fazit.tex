\chapter{Fazit}
Die Justage des optischen Systems um ein Signal sichtbar zu machen war zu Begin sehr umständlich. Nicht nur waren durch
den übrigen Laborbetrieb des öfteren Personen im Strahlweg, auch mussten wir um die sonstige Laboreinrichtung nicht zu
beschädigen mit einem erhöhten Maß an Geschicklichkeit aufwarten. Durch die beschriebene \glqq{}Lücke\grqq{} zwischen
den Labortischen musste der Aufbau zwei mal durchgeführt werden. Hier wäre ein Aufbau entlang einer Laborwand
möglicherweise geschickter. Einerseits würde es sekundäre Komplikationen minimieren und andererseits ermöglicht es eine
kontinuierliche, \glqq{}Lückenlose\grqq{} Reduktion der Strecke.\par
Sonst ist es aber ein schönes Experiment um ein \textit{leibhaftiges} Gefühl dafür zu bekommen, dass Licht sich in der
Tat mit endlicher Geschwindigkeit ausbreitet. Mit Blick auf die gemessenen Werte verblüfft es noch mehr, dass mit
vergleichsweise einfachen Mitteln die gemessene Lichtgeschwindigkeit in bodennaher Luft - durch den relativ geringen
Unterschied sicherlich auch auf eine Messung im Vakuum adaptierbar - mit einer Abweichung von \((5,1 \pm 4,3)\%\) für
\(c_1\) bzw. \((4,7 \pm 1)\%\) für \(c_2\) überraschend nah am rechnerischen Wert von \(\approx \SI{3 \cdot 10^{8}}{\frac{m}{s}}\)
(vgl. \gl{eq:brechung_luft}) liegen.\par\medskip
%
Die ermittelte Brennweite der Linse weicht von der Herstellerangabe deutlich ab, befindet sich allerdings noch im
plausiblen Bereich. Der Fehler der ermittelten Brennweite deckt den Herstellerwert auch nicht ab. Das liegt daran, dass
für den Fehler nur die Abstandsabweichungen der Gegenstands- und Bildweite berücksichtigt wurden, weshalb der Fehler
relativ klein erscheint. Weitere vernachlässigte Abweichungen könnten zum Beispiel sein, dass das Auffangen des
Laserpunkts über ein größeres Intervall der Abstände möglich war, in unserem Fall jedoch nur ein (Abstands-)Punkt zur
Messung gewählt wurde.\par